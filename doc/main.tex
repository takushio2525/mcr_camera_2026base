%% ============================================================
%% mcr_camera_2026base プログラム設計仕様書
%% ============================================================
%% ビルド: latexmk main.tex (.latexmkrc で uplatex + dvipdfmx)
%% ============================================================
\documentclass[a4paper,11pt,dvipdfmx]{jsarticle}

%% ─── パッケージ ───────────────────────────────
\usepackage[dvipdfmx]{graphicx}
\usepackage[dvipdfmx]{xcolor}
\usepackage[dvipdfmx,
  bookmarks=true,
  bookmarksnumbered=true,
  colorlinks=true,
  linkcolor={blue!70!black},
  citecolor={green!50!black},
  urlcolor={blue!60!black},
  pdfauthor={Shiozawa},
  pdftitle={mcr\_camera\_2026base プログラム設計仕様書},
  pdfsubject={マイコンカーラリー カメラクラス開発仕様書},
]{hyperref}
\usepackage{pxjahyper}           % hyperref 日本語しおり対応
\usepackage[top=25mm,bottom=25mm,left=20mm,right=20mm]{geometry}
\usepackage{fancyhdr}
\usepackage{lastpage}
\usepackage{tabularx}
\usepackage{booktabs}
\usepackage{longtable}
\usepackage{enumitem}
\usepackage{titlesec}
\usepackage{tcolorbox}
\tcbuselibrary{skins,breakable}
\usepackage{tikz}
\usetikzlibrary{arrows.meta,positioning,shapes.geometric,fit,calc}
\usepackage{listings}
\usepackage{caption}
\usepackage{float}
\usepackage{amsmath}

%% ─── 色定義 ──────────────────────────────────
\definecolor{headerblue}{HTML}{1B3A5C}
\definecolor{accentblue}{HTML}{2E86C1}
\definecolor{lightgray}{HTML}{F2F3F4}
\definecolor{codebg}{HTML}{F7F9FB}
\definecolor{warnred}{HTML}{E74C3C}
\definecolor{noteyellow}{HTML}{F39C12}
\definecolor{okgreen}{HTML}{27AE60}

%% ─── ヘッダー・フッター ───────────────────────
\pagestyle{fancy}
\fancyhf{}
\fancyhead[L]{\small\color{headerblue}\textbf{mcr\_camera\_2026base} プログラム設計仕様書}
\fancyhead[R]{\small\color{headerblue}Rev.\,1.0}
\fancyfoot[C]{\small\thepage\,/\,\pageref{LastPage}}
\fancyfoot[R]{\small\color{gray}2026-02-19}
\renewcommand{\headrulewidth}{0.8pt}
\renewcommand{\footrulewidth}{0.4pt}

%% ─── セクション見出しスタイル ─────────────────
\titleformat{\section}
  {\Large\bfseries\color{headerblue}}
  {\thesection}{1em}{}
  [\vspace{-0.5em}\textcolor{accentblue}{\rule{\textwidth}{1.2pt}}]
\titleformat{\subsection}
  {\large\bfseries\color{headerblue}}
  {\thesubsection}{1em}{}
\titleformat{\subsubsection}
  {\normalsize\bfseries\color{accentblue}}
  {\thesubsubsection}{1em}{}

%% ─── listings 設定 (C++) ──────────────────────
\lstdefinestyle{cppstyle}{
  language=C++,
  basicstyle=\ttfamily\small,
  keywordstyle=\color{accentblue}\bfseries,
  commentstyle=\color{gray}\itshape,
  stringstyle=\color{okgreen},
  numberstyle=\tiny\color{gray},
  numbers=left,
  numbersep=8pt,
  frame=single,
  framerule=0.5pt,
  rulecolor=\color{accentblue!40},
  backgroundcolor=\color{codebg},
  breaklines=true,
  breakatwhitespace=false,
  tabsize=2,
  showstringspaces=false,
  captionpos=b,
  xleftmargin=15pt,
  framexleftmargin=15pt,
  morekeywords={override,float,int,void,static,const},
}
\lstset{style=cppstyle}

%% ─── tcolorbox スタイル ──────────────────────
% データシート風ボックス
\newtcolorbox{apibox}[1]{
  enhanced,
  colback=white,
  colframe=accentblue,
  coltitle=white,
  fonttitle=\bfseries\large,
  title=#1,
  top=4mm,
  arc=2mm,
  breakable,
  before upper={\parindent0pt},
  attach boxed title to top left={yshift=-3mm,xshift=5mm},
  boxed title style={colback=accentblue,arc=1.5mm},
}

% 注意ボックス
\newtcolorbox{cautionbox}[1][注意]{
  enhanced,
  colback=warnred!5,
  colframe=warnred!70,
  coltitle=white,
  fonttitle=\bfseries,
  title={\raisebox{-0.1em}{\textbf{⚠}} #1},
  arc=1.5mm,
  breakable,
  attach boxed title to top left={yshift=-2mm,xshift=5mm},
  boxed title style={colback=warnred!70,arc=1mm},
}

% ノートボックス
\newtcolorbox{notebox}[1][補足]{
  enhanced,
  colback=accentblue!5,
  colframe=accentblue!50,
  coltitle=white,
  fonttitle=\bfseries,
  title={$\triangleright$ #1},
  arc=1.5mm,
  breakable,
  attach boxed title to top left={yshift=-2mm,xshift=5mm},
  boxed title style={colback=accentblue!50,arc=1mm},
}

% 関数ドキュメント用ボックス (FSPスタイル)
\newtcolorbox{funcblock}[1]{
  enhanced,
  colback=white,
  colframe=accentblue!25,
  coltitle=headerblue,
  fonttitle=\bfseries\ttfamily\small,
  title={◆ #1},
  arc=1mm,
  boxrule=0.4pt,
  breakable,
  left=3mm, right=3mm, top=2mm, bottom=2mm,
  before skip=3mm, after skip=3mm,
  before upper={\parindent0pt},
}

% APIセクション見出し (FSPスタイル)
\newcommand{\apisection}[1]{%
  \vspace{4mm}%
  {\normalsize\bfseries\color{headerblue} #1}%
  \par\vspace{-1mm}%
  {\color{accentblue!40}\rule{\linewidth}{0.5pt}}%
  \par\vspace{2mm}%
}

% 実行フェーズバッジ
\newcommand{\phasebadge}[1]{%
  \colorbox{accentblue!10}{\small\bfseries\color{accentblue!80!black}\,#1\,}%
}

%% ─── ドキュメント情報 ─────────────────────────
\newcommand{\doctitle}{mcr\_camera\_2026base\\プログラム設計仕様書}
\newcommand{\docsubtitle}{GR-PEACH (RZ/A1H) マイコンカーラリー カメラクラス開発}
\newcommand{\docauthor}{Shiozawa}
\newcommand{\docversion}{1.0}
\newcommand{\docdate}{2026年2月19日}

%% ============================================================
\begin{document}

%% ─── 表紙 ─────────────────────────────────────
\begin{titlepage}
\begin{center}
\vspace*{30mm}

{\color{accentblue}\rule{\textwidth}{2pt}}
\vspace{8mm}

{\Huge\bfseries\color{headerblue} \doctitle}

\vspace{6mm}

{\Large\color{accentblue} \docsubtitle}

\vspace{4mm}
{\color{accentblue}\rule{\textwidth}{2pt}}

\vspace{20mm}

\begin{tabular}{rl}
  \textbf{バージョン} & \docversion \\[4pt]
  \textbf{作成日}     & \docdate \\[4pt]
  \textbf{作成者}     & \docauthor \\[4pt]
  \textbf{ステータス} & ドラフト \\
\end{tabular}

\vfill

\begin{tcolorbox}[
  colback=lightgray,
  colframe=headerblue,
  width=0.8\textwidth,
  arc=3mm,
  halign=center,
]
\small
本文書は \texttt{mcr\_camera\_2026base} プロジェクトのプログラム設計仕様書です。\\
ベアメタル環境における割り込み駆動アーキテクチャの詳細設計を記述します。
\end{tcolorbox}

\vspace{10mm}
{\small\color{gray} 機密区分: 内部利用}

\end{center}
\end{titlepage}

%% ─── 改訂履歴 ─────────────────────────────────
\newpage
\section*{改訂履歴}
\addcontentsline{toc}{section}{改訂履歴}

\begin{tabularx}{\textwidth}{c c c X}
\toprule
\textbf{版} & \textbf{日付} & \textbf{著者} & \textbf{変更内容} \\
\midrule
1.0 & 2026-02-19 & Shiozawa & 初版作成 \\
\bottomrule
\end{tabularx}

%% ─── 目次 ─────────────────────────────────────
\newpage
\tableofcontents
\newpage

%% ============================================================
%% 1. 概要
%% ============================================================
\section{概要}\label{sec:overview}

GR-PEACH (RZ/A1H) を使用した、マイコンカーラリー (MCR) カメラクラス開発のためのベースプログラムである。

\begin{notebox}[設計思想]
\begin{itemize}[nosep]
  \item mbed OS を使用せず、ルネサス標準の \texttt{iodefine.h} を用いたベアメタル(レジスタ直接操作)環境で構築する
  \item 既存のスパゲッティコードを廃し、クラスベースの \texttt{init}/\texttt{update} パターンを採用したモダンなアーキテクチャを目指す
\end{itemize}
\end{notebox}

\subsection{用語定義}

\begin{tabularx}{\textwidth}{l X}
\toprule
\textbf{用語} & \textbf{定義} \\
\midrule
MCR       & マイコンカーラリーの略称 \\
GR-PEACH  & ルネサス RZ/A1H 搭載の開発ボード \\
MTU2      & マルチファンクション・タイマ・ユニット2。PWM出力に使用 \\
ラッチ方式 & セッターで値を内部変数に保持し、\texttt{update()} 呼び出し時にのみハードウェアへ反映する方式 \\
IModule   & 全モジュール共通の基底インターフェース (\hyperref[sec:api-imodule]{§\ref*{sec:api-imodule}}) \\
\bottomrule
\end{tabularx}

%% ============================================================
%% 2. 背景・課題
%% ============================================================
\section{背景・課題}\label{sec:background}

\begin{cautionbox}[開発環境の課題]
\begin{enumerate}[nosep]
  \item \textbf{mbed OS サービス終了}: mbed のサービス終了に伴い、将来的な継続利用や保守が困難になる
  \item \textbf{開発環境の老朽化}: 既存プロジェクトの e2 studio 環境やコンパイラバージョンが古く、最新の環境でビルドを通す必要がある
  \item \textbf{ブラックボックス化の解消}: OS内部処理への依存を減らし、デバッグを容易にする必要がある
\end{enumerate}
\end{cautionbox}

%% ============================================================
%% 3. 目的
%% ============================================================
\section{目的}\label{sec:purpose}

\begin{enumerate}[nosep]
  \item mbed OS に依存しない、RZ/A1H 用の C++ ベースプログラム環境の構築
  \item \textbf{クラス化・モジュール化}による可読性・保守性の向上
  \item 共通インターフェース (\texttt{IModule}) による統一的な制御フローの確立
\end{enumerate}

%% ============================================================
%% 4. システムアーキテクチャ
%% ============================================================
\section{システムアーキテクチャ}\label{sec:architecture}

システムはタイマー割り込みによる定期実行(1ループ/1ms を想定)で完全に同期して動作する。
処理は3つのフェーズに分離される。

\subsection{実行フロー図}\label{sec:flow}

\begin{figure}[H]
\centering
\begin{tikzpicture}[
  node distance=6mm and 10mm,
  phase/.style={
    draw=accentblue, fill=accentblue!10,
    rounded corners=3pt, minimum width=55mm, minimum height=10mm,
    font=\bfseries\small, align=center,
  },
  module/.style={
    draw=headerblue, fill=headerblue!8,
    rounded corners=2pt, minimum width=40mm, minimum height=8mm,
    font=\small\ttfamily, align=center,
  },
  arr/.style={-{Stealth[length=3mm]}, thick, color=accentblue},
]

% フェーズ
\node[phase] (input) {Input Phase};
\node[phase, below=15mm of input] (logic) {Logic Phase (Pattern)};
\node[phase, below=15mm of logic] (output) {Output Phase};

% Input モジュール
\node[module, right=12mm of input, yshift=4mm]  (enc) {Encoder.update()};
\node[module, right=12mm of input, yshift=-4mm] (cam) {Camera.update()};

% Logic
\node[module, right=12mm of logic] (pat) {判断 → set...()};

% Output モジュール
\node[module, right=12mm of output, yshift=9mm]  (mot) {Motor.update()};
\node[module, right=12mm of output, yshift=0mm]  (srv) {Servo.update()};
\node[module, right=12mm of output, yshift=-9mm] (onb) {Onboard.update()};

% 矢印
\draw[arr] (input) -- (logic);
\draw[arr] (logic) -- (output);

\draw[arr] (input.east) ++(0,4mm) -- (enc.west);
\draw[arr] (input.east) ++(0,-4mm) -- (cam.west);

\draw[arr] (logic.east) -- (pat.west);

\draw[arr] (output.east) ++(0,9mm) -- (mot.west);
\draw[arr] (output.east) -- (srv.west);
\draw[arr] (output.east) ++(0,-9mm) -- (onb.west);

\end{tikzpicture}
\caption{割り込みハンドラ実行フロー}
\label{fig:flow}
\end{figure}

\subsection{3フェーズモデル}

\begin{tabularx}{\textwidth}{c l X}
\toprule
\textbf{順序} & \textbf{フェーズ} & \textbf{内容} \\
\midrule
1 & Input Phase  & センサ値の読み取り(Encoder, Camera) \\
2 & Logic Phase  & 判断ロジック実行、各ドライバの \texttt{set...()} を呼び出し(\textbf{この時点では出力しない}) \\
3 & Output Phase & 全ドライバの \texttt{update()} を呼び出し、ハードウェアへ反映 \\
\bottomrule
\end{tabularx}

\begin{notebox}[ラッチ方式の利点]
Logic Phase でセットした値は即座にハードウェアへ反映されない。
Output Phase で一括反映することで、割り込み内での出力タイミングを統一し、
グリッチや不整合を防止する。
\end{notebox}

%% ============================================================
%% 5. ファイル構成
%% ============================================================
\section{ファイル構成}\label{sec:files}

\subsection{ディレクトリ構造}

\begin{tcolorbox}[colback=codebg, colframe=accentblue!40, arc=2mm]
\ttfamily\small
\begin{tabbing}
\hspace{4mm}\=\hspace{4mm}\=\hspace{4mm}\=\hspace{4mm}\=\kill
mcr\_camera\_test1/ \\
\> src/ \\
\>\> mcr\_camera\_test1.cpp \quad\textrm{\color{gray}← メイン・割り込みハンドラ} \\
\>\> core/ \\
\>\>\> IModule.h \quad\textrm{\color{gray}← 基底インターフェース} \\
\>\> drivers/ \\
\>\>\> Camera.h / Camera.cpp \quad\textrm{\color{gray}← カメラ制御} \\
\>\>\> Encoder.h / Encoder.cpp \quad\textrm{\color{gray}← エンコーダ制御} \\
\>\>\> Motor.h / Motor.cpp \quad\textrm{\color{gray}← モータ制御} \\
\>\>\> Onboard.h / Onboard.cpp \quad\textrm{\color{gray}← LED・スイッチ制御} \\
\>\>\> SDCard.h / SDCard.cpp \quad\textrm{\color{gray}← SDカード制御} \\
\>\>\> Serial.h / Serial.cpp \quad\textrm{\color{gray}← シリアル通信} \\
\>\>\> Servo.h / Servo.cpp \quad\textrm{\color{gray}← サーボ制御} \\
\>\> modules/ \\
\>\>\> Trace.h / Trace.cpp \quad\textrm{\color{gray}← ライントレースロジック} \\
\>\> utils/ \quad\textrm{\color{gray}← ユーティリティ (予定)} \\
\> generate/ \\
\>\> iodefine.h \quad\textrm{\color{gray}← レジスタ定義 (自動生成)} \\
\> doc/ \\
\>\> main.tex \quad\textrm{\color{gray}← 本仕様書} \\
\end{tabbing}
\end{tcolorbox}

\subsection{ディレクトリ役割}

\begin{tabularx}{\textwidth}{l X}
\toprule
\textbf{ディレクトリ} & \textbf{役割} \\
\midrule
\texttt{src/core/}    & システム基盤(\texttt{IModule} インターフェース) \\
\texttt{src/modules/} & ロジックモジュール(\texttt{Trace} 等 --- 未実装) \\
\texttt{src/drivers/} & ハードウェアドライバ(Motor, Servo, Encoder, Camera, Serial, SDCard, Onboard) \\
\texttt{src/utils/}   & ユーティリティ関数 \\
\texttt{generate/}    & e2 studio 自動生成ファイル(\texttt{iodefine.h} 等) \\
\texttt{doc/}         & ドキュメント \\
\bottomrule
\end{tabularx}

\subsection{全ファイル一覧}

\begin{longtable}{l p{95mm}}
\toprule
\textbf{ファイル} & \textbf{役割} \\
\midrule
\endhead
\texttt{src/mcr\_camera\_2026base.cpp}    & メインエントリ・割り込みハンドラ・Pattern ロジック \\
\texttt{src/core/IModule.h}            & 全モジュール共通の基底インターフェース \\
\texttt{src/drivers/Motor.h/.cpp}      & モータPWM制御 (MTU2 ch3/4) \\
\texttt{src/drivers/Servo.h/.cpp}      & サーボPWM制御 (MTU2 ch0) \\
\texttt{src/drivers/Encoder.h/.cpp}    & ロータリーエンコーダ読取 (MTU2 位相計数) \\
\texttt{src/drivers/Camera.h/.cpp}     & カメラ画像取得・ライン検出 (CEU/DMA) \\
\texttt{src/drivers/Serial.h/.cpp}     & シリアル通信 (SCIF) \\
\texttt{src/drivers/SDCard.h/.cpp}     & SDカードログ書込み (SPI) \\
\texttt{src/drivers/Onboard.h/.cpp}    & オンボードLED・スイッチ制御 (GPIO) \\
\texttt{src/modules/Trace.h/.cpp}      & ライントレース・PID制御ロジック \\
\bottomrule
\end{longtable}

\subsection{機能とファイルの対応表}

\begin{tabularx}{\textwidth}{l l X}
\toprule
\textbf{機能} & \textbf{メインファイル} & \textbf{関連ファイル} \\
\midrule
割り込みハンドラ & \texttt{mcr\_camera\_2026base.cpp} & 全ドライバ・モジュール \\
モータ制御       & \texttt{Motor.cpp}       & IModule.h, iodefine.h \\
サーボ制御       & \texttt{Servo.cpp}       & IModule.h, iodefine.h \\
エンコーダ       & \texttt{Encoder.cpp}     & IModule.h, iodefine.h \\
カメラ           & \texttt{Camera.cpp}      & IModule.h, iodefine.h \\
シリアル通信     & \texttt{Serial.cpp}      & IModule.h, iodefine.h \\
SDカード         & \texttt{SDCard.cpp}      & IModule.h, iodefine.h \\
LED・スイッチ    & \texttt{Onboard.cpp}     & IModule.h, iodefine.h \\
ライントレース   & \texttt{Trace.cpp}       & IModule.h, Camera.h, Encoder.h \\
\bottomrule
\end{tabularx}

%% ============================================================
%% 6. 技術スタック
%% ============================================================
\section{技術スタック}\label{sec:techstack}

\begin{tabularx}{\textwidth}{l X}
\toprule
\textbf{項目} & \textbf{内容} \\
\midrule
ターゲットボード & GR-PEACH (ルネサス RZ/A1H, ARM Cortex-A9) \\
開発環境         & Renesas e2 studio \\
言語             & C++ (ベアメタル、mbed OS 不使用) \\
コンパイラ       & GCC ARM (e2 studio 標準) \\
レジスタ定義     & \texttt{iodefine.h} (e2 studio 自動生成) \\
\bottomrule
\end{tabularx}

%% ============================================================
%% 7. API仕様 (データシート)
%% ============================================================
\section{API仕様}\label{sec:api}

本章では各クラスの公開APIをデータシート形式で記述する。
全クラスは \texttt{IModule} インターフェース (\hyperref[sec:api-imodule]{§\ref*{sec:api-imodule}}) を継承する。

%% ── 7.1 IModule ──
\subsection{IModule インターフェース}\label{sec:api-imodule}

\begin{apibox}{IModule --- 基底インターフェース}

\textbf{ファイル:} \texttt{src/core/IModule.h} \hfill
\textbf{種別:} 純粋仮想クラス

\vspace{2mm}\hrule\vspace{2mm}

\apisection{概要}

全てのドライバおよびロジックモジュールが実装すべき共通インターフェース。
\texttt{init()} で初期化、\texttt{update()} で周期処理を行う。

\apisection{機能一覧}

\begin{itemize}[nosep]
  \item 全モジュール共通の初期化・周期処理メソッドを定義
  \item 仮想デストラクタによる安全なポリモーフィズム
\end{itemize}

\apisection{関数ドキュメント}

\begin{funcblock}{init()}
\texttt{virtual void init() = 0}

\vspace{1.5mm}
初期化処理。レジスタ設定・変数初期化を行う。システム起動時に1回のみ呼び出される。
\end{funcblock}

\begin{funcblock}{update()}
\texttt{virtual void update() = 0}

\vspace{1.5mm}
周期処理。割り込みハンドラまたはメインループから定期的に呼び出される。
\end{funcblock}

\apisection{実装例}

\begin{lstlisting}[caption={IModule インターフェース定義},label=lst:imodule]
class IModule {
public:
  virtual ~IModule() {}
  virtual void init() = 0;   // 初期化処理
  virtual void update() = 0; // 周期処理
};
\end{lstlisting}

\end{apibox}

%% ── 7.2 Encoder ──
\subsection{Encoder クラス}\label{sec:api-encoder}

\begin{apibox}{Encoder --- エンコーダドライバ}

\textbf{ファイル:} \texttt{src/drivers/Encoder.h}, \texttt{Encoder.cpp} \hfill
\textbf{継承:} \texttt{IModule} \hfill
\phasebadge{Input Phase}

\vspace{2mm}\hrule\vspace{2mm}

\apisection{概要}

MTU2 (チャネル1) の位相計数モード2を使用し、エンコーダのパルスをカウントする。
フィルタリング処理を行い、走行距離と現在の速度を算出・管理する。

\apisection{機能一覧}

\begin{itemize}[nosep]
  \item トータル走行距離・コース走行距離の管理
  \item 現在速度の算出
  \item 距離の任意設定(クリア機能含む)
\end{itemize}

\apisection{内部定数}

\begin{tabularx}{\textwidth}{l l X}
\toprule
\textbf{定数名} & \textbf{値} & \textbf{説明} \\
\midrule
\texttt{Filter\_N} & 2 & フィルタバッファサイズ \\
\texttt{DIST\_PER\_PULSE} & (定数) & 1パルスあたりの移動距離 [mm] \\
\bottomrule
\end{tabularx}

\apisection{内部変数}

\begin{tabularx}{\textwidth}{l l l X}
\toprule
\textbf{変数名} & \textbf{型} & \textbf{初期値} & \textbf{説明} \\
\midrule
\texttt{\_totalDistance}  & \texttt{float} & 0.0 & トータル走行距離 \\
\texttt{\_courseDistance} & \texttt{float} & 0.0 & コース走行距離 \\
\texttt{\_speed}          & \texttt{float} & 0.0 & 現在速度 \\
\texttt{\_deltaCount}     & \texttt{int}   & 0   & 今回のパルス変動量 \\
\texttt{\_filter\_buf[]}  & \texttt{int[]} & 0   & フィルタ用バッファ \\
\bottomrule
\end{tabularx}

\apisection{関数ドキュメント}

\begin{funcblock}{init()}
\texttt{void init(void)}

\vspace{1.5mm}
MTU2の位相計数モード2を初期化する。GPIO設定およびタイマスタートを行う。
\end{funcblock}

\begin{funcblock}{update()}
\texttt{void update(void)}

\vspace{1.5mm}
MTU2カウンタ値を読み取り、前回値との差分から距離・速度を更新する。
\end{funcblock}

\begin{funcblock}{getTotalDistance()}
\texttt{float getTotalDistance(void)} --- トータル走行距離 (\texttt{\_totalDistance}) を返す。
\end{funcblock}

\begin{funcblock}{getCourseDistance()}
\texttt{float getCourseDistance(void)} --- コース走行距離 (\texttt{\_courseDistance}) を返す。
\end{funcblock}

\begin{funcblock}{getSpeed()}
\texttt{float getSpeed(void)} --- 現在速度 (\texttt{\_speed}) を返す。
\end{funcblock}

\begin{funcblock}{setTotalDistance(dist)}
\texttt{void setTotalDistance(float dist)} --- トータル走行距離を任意の値に設定する (0でクリア)。
\end{funcblock}

\begin{funcblock}{setCourseDistance(dist)}
\texttt{void setCourseDistance(float dist)} --- コース走行距離を任意の値に設定する (0でクリア)。
\end{funcblock}

\end{apibox}

%% ── 7.3 Camera ──
\subsection{Camera クラス}\label{sec:api-camera}

\begin{apibox}{Camera --- カメラ制御および画像処理ドライバ}

\textbf{ファイル:} \texttt{src/drivers/Camera.h}, \texttt{Camera.cpp} \hfill
\textbf{継承:} \texttt{IModule} \hfill
\phasebadge{Input Phase}

\vspace{2mm}\hrule\vspace{2mm}

\apisection{概要}


\apisection{関数ドキュメント (画像処理)}

\begin{funcblock}{resize(src, dst, percent)}
\texttt{void resize(uint8\_t *src, uint8\_t *dst, double percent)}

\vspace{1.5mm}
(旧: \texttt{ImageReduction}) --- 画像を縮小する。
\end{funcblock}

\begin{funcblock}{extractBrightness(src, dst)}
\texttt{void extractBrightness(uint8\_t *src, uint8\_t *dst)}

\vspace{1.5mm}
(旧: \texttt{Extraction\_Brightness}) --- YCbCr画像から輝度成文(Y)を抽出する。
\end{funcblock}

\begin{funcblock}{binarize(src, dst, threshold)}
\texttt{void binarize(uint8\_t *src, uint8\_t *dst, int threshold)}

\vspace{1.5mm}
(旧: \texttt{Binarization}) --- 画像を指定閾値で二値化する。
\end{funcblock}

\begin{funcblock}{percentileMethod(src, percent)}
\texttt{int percentileMethod(uint8\_t *src, int percent)}

\vspace{1.5mm}
(旧: \texttt{Percentile\_Method}) --- パーセンタイル法による閾値決定を行う。
\end{funcblock}

\begin{funcblock}{discriminantAnalysis(src)}
\texttt{int discriminantAnalysis(uint8\_t *src)}

\vspace{1.5mm}
(旧: \texttt{DiscriminantAnalysis\_Method}) --- 判別分析法による閾値決定を行う。
\end{funcblock}

\begin{funcblock}{extractPart(src, x, y, dst, w, h)}
\texttt{void extractPart(uint8\_t *src, int x, int y, uint8\_t *dst, int w, int h)}

\vspace{1.5mm}
(旧: \texttt{Image\_part\_Extraction}) --- 画像の一部を切り出す。
\end{funcblock}

\begin{funcblock}{standardDeviation(...)}
\texttt{double standardDeviation(uint8\_t *data, double *devi, int w, int h)}

\vspace{1.5mm}
(旧: \texttt{Standard\_Deviation}) --- 標準偏差を算出する。
\end{funcblock}

\begin{funcblock}{covariance(...)}
\texttt{double covariance(double *deviA, double *deviB, int w, int h)}

\vspace{1.5mm}
(旧: \texttt{Covariance}) --- 共分散を算出する。
\end{funcblock}

\begin{funcblock}{matchPattern(...)}
\texttt{void matchPattern(uint8\_t *src, ImagePartPattern *tmplt, ...)}

\vspace{1.5mm}
(旧: \texttt{PatternMatching\_process}) --- パターンマッチングを行う。
\end{funcblock}

\begin{cautionbox}[実装上の注意]
\begin{itemize}[nosep]
  \item \texttt{DisplayBase} ライブラリのセットアップ手順は複雑なため、既存実装 (\texttt{init\_Camera}) を参考に移植する。
  \item 画像処理関数は重いため、処理時間オーバーに注意が必要。
\end{itemize}
\end{cautionbox}

\end{apibox}

%% ── 7.4 Trace ──
\subsection{Trace クラス}\label{sec:api-trace}

\begin{apibox}{Trace --- ライントレースモジュール}

\textbf{ファイル:} \texttt{src/modules/Trace.h}, \texttt{Trace.cpp} \hfill
\textbf{継承:} \texttt{IModule} \hfill
\phasebadge{Logic Phase}

\vspace{2mm}\hrule\vspace{2mm}

\apisection{概要}

カメラ画像からライン位置を検出し、PID制御によりステアリング角度を算出する。
\texttt{update()} 内で検出→PID計算の一連処理を実行する。

\apisection{機能一覧}

\begin{itemize}[nosep]
  \item 画像二値化によるライン検出
  \item PID制御によるステアリング角度算出
  \item ゲインパラメータの動的設定
  \item ステアリング角度のリミット処理
\end{itemize}

\apisection{内部定数}

\begin{tabularx}{\textwidth}{l l X}
\toprule
\textbf{定数名} & \textbf{値} & \textbf{説明} \\
\midrule
\texttt{THRESHOLD}   & 未定 & 二値化閾値 \\
\texttt{STEER\_MAX}  & 未定 & ステアリング角度上限 \\
\texttt{STEER\_MIN}  & 未定 & ステアリング角度下限 \\
\bottomrule
\end{tabularx}

\apisection{内部変数}

\begin{tabularx}{\textwidth}{l l l X}
\toprule
\textbf{変数名} & \textbf{型} & \textbf{初期値} & \textbf{説明} \\
\midrule
\texttt{\_kp}        & \texttt{float} & \texttt{0.0f}  & 比例ゲイン \\
\texttt{\_ki}        & \texttt{float} & \texttt{0.0f}  & 積分ゲイン \\
\texttt{\_kd}        & \texttt{float} & \texttt{0.0f}  & 微分ゲイン \\
\texttt{\_integral}  & \texttt{float} & \texttt{0.0f}  & 積分値 \\
\texttt{\_prevErr}   & \texttt{float} & \texttt{0.0f}  & 前回偏差 \\
\texttt{\_steer}     & \texttt{float} & \texttt{0.0f}  & 算出ステアリング角度 \\
\texttt{\_linePos}   & \texttt{float} & \texttt{0.0f}  & 検出ライン位置 \\
\texttt{\_lineFound} & \texttt{bool}  & \texttt{false} & ライン検出成否 \\
\bottomrule
\end{tabularx}

\apisection{関数ドキュメント}

\begin{funcblock}{init()}
\texttt{void init(void)}

\vspace{1.5mm}
PIDパラメータおよび内部状態変数を初期化する。
\end{funcblock}

\begin{funcblock}{update()}
\texttt{void update(void)}

\vspace{1.5mm}
ライン検出→PID計算の一連処理を実行する。
内部で \texttt{binarize()}→\texttt{detectLine()}→\texttt{calcPid()}→\texttt{clampSteer()} の順に呼び出す。
\end{funcblock}

\begin{funcblock}{getSteer()}
\texttt{float getSteer(void)} --- 算出されたステアリング角度を返す。
\end{funcblock}

\begin{funcblock}{getLinePos()}
\texttt{float getLinePos(void)} --- 検出されたライン位置を返す。
\end{funcblock}

\begin{funcblock}{isLineFound()}
\texttt{bool isLineFound(void)} --- ラインが検出されたかどうかを返す。
\end{funcblock}

\begin{funcblock}{setKp(v)}
\texttt{void setKp(float v)} --- 比例ゲインを設定する。($v \geq 0$)
\end{funcblock}

\begin{funcblock}{setKi(v)}
\texttt{void setKi(float v)} --- 積分ゲインを設定する。($v \geq 0$)
\end{funcblock}

\begin{funcblock}{setKd(v)}
\texttt{void setKd(float v)} --- 微分ゲインを設定する。($v \geq 0$)
\end{funcblock}

\begin{funcblock}{detectLine(data) \normalfont\small\color{gray}[内部]}
\texttt{float detectLine(uint8\_t* data)}

\vspace{1.5mm}
二値化済みの画像データからライン中心位置を算出する。

\vspace{1.5mm}
\textbf{引数:} \texttt{data} (\texttt{uint8\_t*}) --- 二値化済み画像データ

\textbf{戻り値:} \texttt{float} --- ライン中心位置
\end{funcblock}

\begin{funcblock}{calcPid(error) \normalfont\small\color{gray}[内部]}
\texttt{float calcPid(float error)}

\vspace{1.5mm}
PID制御量を計算する。積分値(\texttt{\_integral}) と前回偏差(\texttt{\_prevErr}) を内部で更新する。

\vspace{1.5mm}
\textbf{引数:} \texttt{error} (\texttt{float}) --- 目標位置との偏差

\textbf{戻り値:} \texttt{float} --- PID制御量
\end{funcblock}

\begin{funcblock}{binarize(data) \normalfont\small\color{gray}[内部]}
\texttt{void binarize(uint8\_t* data)}

\vspace{1.5mm}
画像データを閾値 (\texttt{THRESHOLD}) に基づいて二値化する。入力バッファを直接書き換える。

\vspace{1.5mm}
\textbf{引数:} \texttt{data} (\texttt{uint8\_t*}) --- 画像データ(in-place変換)
\end{funcblock}

\begin{funcblock}{clampSteer(val) \normalfont\small\color{gray}[内部]}
\texttt{float clampSteer(float val)}

\vspace{1.5mm}
ステアリング角度を \texttt{STEER\_MIN}〜\texttt{STEER\_MAX} の範囲にクランプする。

\vspace{1.5mm}
\textbf{引数:} \texttt{val} (\texttt{float}) --- クランプ前のステアリング角度

\textbf{戻り値:} \texttt{float} --- クランプ後のステアリング角度
\end{funcblock}

\end{apibox}

%% ── 7.5 Motor ──
\subsection{Motor クラス}\label{sec:api-motor}

\begin{apibox}{Motor --- モータ PWM ドライバ}

\textbf{ファイル:} \texttt{src/drivers/Motor.h}, \texttt{Motor.cpp} \hfill
\textbf{継承:} \texttt{IModule} \hfill
\phasebadge{Output Phase}

\vspace{2mm}\hrule\vspace{2mm}

\apisection{概要}

MTU2 チャネル3/4を使用したリセット同期PWMモードで左右モータを制御する。
ラッチ方式を採用し、\texttt{setPower()} でデューティ比を内部保持、\texttt{update()} でレジスタへ反映する。

\apisection{機能一覧}

\begin{itemize}[nosep]
  \item MTU2 リセット同期PWMモードによるモータ駆動
  \item 左右独立のデューティ比制御 ($-1.0 \sim 1.0$)
  \item 符号による正転/逆転の方向制御
  \item ラッチ方式による安全な出力タイミング
\end{itemize}

\apisection{内部定数}

\begin{tabularx}{\textwidth}{l l X}
\toprule
\textbf{定数} & \textbf{値} & \textbf{説明} \\
\midrule
\texttt{PWM\_CYCLE} & 33332 & PWM周期カウント値 (1ms @ P0$\phi$/1) \\
\bottomrule
\end{tabularx}

\apisection{内部変数}

\begin{tabularx}{\textwidth}{l l l X}
\toprule
\textbf{変数名} & \textbf{型} & \textbf{初期値} & \textbf{説明} \\
\midrule
\texttt{\_pwmLeft}  & \texttt{float} & \texttt{0.0f} & 左モータデューティ比 ($-1.0 \sim 1.0$) \\
\texttt{\_pwmRight} & \texttt{float} & \texttt{0.0f} & 右モータデューティ比 ($-1.0 \sim 1.0$) \\
\bottomrule
\end{tabularx}

\apisection{関数ドキュメント}

\begin{funcblock}{init()}
\texttt{void init(void)}

\vspace{1.5mm}
GPIO方向ピンおよびMTU2チャネル3/4のレジスタを初期化する。
リセット同期PWMモードを設定し、カウント動作を開始する。
\end{funcblock}

\begin{funcblock}{update()}
\texttt{void update(void)}

\vspace{1.5mm}
内部変数のデューティ比をハードウェアへ反映する。
内部で \texttt{setDirection()}→\texttt{calcDuty()}→\texttt{applyPwm()} の順に呼び出す。
\end{funcblock}

\begin{funcblock}{setPower(l, r)}
\texttt{void setPower(float l, float r)}

\vspace{1.5mm}
左右モータのデューティ比を内部変数に保存する(ラッチ方式)。
この時点ではハードウェアへの反映は行わない。

\vspace{1.5mm}
\begin{itemize}[nosep, leftmargin=*]
  \item \texttt{l} (\texttt{float}) --- 左モータデューティ比 ($-1.0 \sim 1.0$、負値で逆転)
  \item \texttt{r} (\texttt{float}) --- 右モータデューティ比 ($-1.0 \sim 1.0$、負値で逆転)
\end{itemize}
\end{funcblock}

\begin{funcblock}{calcDuty(pwm) \normalfont\small\color{gray}[内部]}
\texttt{uint16\_t calcDuty(float pwm)}

\vspace{1.5mm}
デューティ比(浮動小数点数)をPWMカウント値に変換する。
$\mathit{duty} = \lfloor |\mathit{pwm}| \times \mathtt{PWM\_CYCLE} \rfloor$

\vspace{1.5mm}
\textbf{引数:} \texttt{pwm} (\texttt{float}) --- デューティ比 ($-1.0 \sim 1.0$)

\textbf{戻り値:} \texttt{uint16\_t} --- PWMカウント値
\end{funcblock}

\begin{funcblock}{setDirection(l, r) \normalfont\small\color{gray}[内部]}
\texttt{void setDirection(float l, float r)}

\vspace{1.5mm}
デューティ比の符号に基づき、方向ピン (GPIO P4\_6, P4\_7) を設定する。
正値で正転 (Low)、負値で逆転 (High)。

\vspace{1.5mm}
\begin{itemize}[nosep, leftmargin=*]
  \item \texttt{l} (\texttt{float}) --- 左モータデューティ比
  \item \texttt{r} (\texttt{float}) --- 右モータデューティ比
\end{itemize}
\end{funcblock}

\begin{funcblock}{applyPwm(l, r) \normalfont\small\color{gray}[内部]}
\texttt{void applyPwm(uint16\_t l, uint16\_t r)}

\vspace{1.5mm}
MTU2レジスタ (TGRA\_4/TGRC\_4, TGRB\_4/TGRD\_4) へPWM値を書き込む。

\vspace{1.5mm}
\begin{itemize}[nosep, leftmargin=*]
  \item \texttt{l} (\texttt{uint16\_t}) --- 左モータPWMカウント値
  \item \texttt{r} (\texttt{uint16\_t}) --- 右モータPWMカウント値
\end{itemize}
\end{funcblock}

\apisection{ハードウェアマッピング}

\begin{tabularx}{\textwidth}{l l l X}
\toprule
\textbf{信号} & \textbf{ピン} & \textbf{レジスタ} & \textbf{説明} \\
\midrule
左モータ PWM  & P4\_4 (TIOC4A) & TGRA\_4 / TGRC\_4 & デューティ比 \\
右モータ PWM  & P4\_5 (TIOC4B) & TGRB\_4 / TGRD\_4 & デューティ比 \\
左モータ 方向 & P4\_6 (GPIO)   & GPIO.P4 bit6      & Low:正転 / High:逆転 \\
右モータ 方向 & P4\_7 (GPIO)   & GPIO.P4 bit7      & Low:正転 / High:逆転 \\
\bottomrule
\end{tabularx}

\end{apibox}

%% ── 7.6 Servo ──
\subsection{Servo クラス}\label{sec:api-servo}

\begin{apibox}{Servo --- サーボ PWM ドライバ}

\textbf{ファイル:} \texttt{src/drivers/Servo.h}, \texttt{Servo.cpp} \hfill
\textbf{継承:} \texttt{IModule} \hfill
\phasebadge{Output Phase}

\vspace{2mm}\hrule\vspace{2mm}

\apisection{概要}

MTU2 チャネル0をPWMモード1で駆動し、ステアリングサーボの角度を制御する。
ラッチ方式を採用し、\texttt{setAngle()} で角度を内部保持、\texttt{update()} でレジスタへ反映する。

\apisection{機能一覧}

\begin{itemize}[nosep]
  \item MTU2 PWMモード1によるサーボ駆動
  \item 角度→パルス幅の自動変換
  \item ラッチ方式による安全な出力タイミング
\end{itemize}

\apisection{内部定数}

\begin{tabularx}{\textwidth}{l l X}
\toprule
\textbf{定数名} & \textbf{値} & \textbf{説明} \\
\midrule
\texttt{SERVO\_CENTER} & 3090 & サーボ中心位置のパルス幅カウント値 \\
\texttt{HANDLE\_STEP}  & 23   & 1度あたりのカウント値 \\
\bottomrule
\end{tabularx}

\apisection{内部変数}

\begin{tabularx}{\textwidth}{l l l X}
\toprule
\textbf{変数名} & \textbf{型} & \textbf{初期値} & \textbf{説明} \\
\midrule
\texttt{\_angle} & \texttt{float} & \texttt{0.0f} & サーボ角度 (度) \\
\bottomrule
\end{tabularx}

\apisection{関数ドキュメント}

\begin{funcblock}{init()}
\texttt{void init(void)}

\vspace{1.5mm}
GPIO および MTU2チャネル0のレジスタを初期化する。PWMモード1を設定し、カウント動作を開始する。
\end{funcblock}

\begin{funcblock}{update()}
\texttt{void update(void)}

\vspace{1.5mm}
内部変数の角度値をハードウェアへ反映する。
内部で \texttt{calcPulse()}→\texttt{applyServo()} の順に呼び出す。
\end{funcblock}

\begin{funcblock}{setAngle(angle)}
\texttt{void setAngle(float angle)}

\vspace{1.5mm}
サーボ角度を内部変数に保存する(ラッチ方式)。0が中心、正負で左右に振れる。

\vspace{1.5mm}
\textbf{引数:} \texttt{angle} (\texttt{float}) --- サーボ角度 (度、機体依存)
\end{funcblock}

\begin{funcblock}{calcPulse(angle) \normalfont\small\color{gray}[内部]}
\texttt{uint16\_t calcPulse(float angle)}

\vspace{1.5mm}
角度をパルス幅カウント値に変換する。
$\mathit{val} = \lfloor \mathit{angle} \times 23 \rfloor + \mathtt{SERVO\_CENTER}$

\vspace{1.5mm}
\textbf{引数:} \texttt{angle} (\texttt{float}) --- サーボ角度 (度)

\textbf{戻り値:} \texttt{uint16\_t} --- パルス幅カウント値
\end{funcblock}

\begin{funcblock}{applyServo(val) \normalfont\small\color{gray}[内部]}
\texttt{void applyServo(uint16\_t val)}

\vspace{1.5mm}
MTU2レジスタ (TGRB\_0 / TGRD\_0) へパルス幅を書き込む。

\vspace{1.5mm}
\textbf{引数:} \texttt{val} (\texttt{uint16\_t}) --- パルス幅カウント値
\end{funcblock}

\apisection{ハードウェアマッピング}

\begin{tabularx}{\textwidth}{l l l X}
\toprule
\textbf{信号} & \textbf{ピン} & \textbf{レジスタ} & \textbf{説明} \\
\midrule
サーボ PWM & P4\_0 (TIOC0A) & TGRB\_0 / TGRD\_0 & パルス幅 (立ち下がりエッジ) \\
\bottomrule
\end{tabularx}

\end{apibox}

%% ── 7.7 Onboard ──
\subsection{Onboard クラス}\label{sec:api-onboard}

\begin{apibox}{Onboard --- オンボード LED / スイッチ ドライバ}

\textbf{ファイル:} \texttt{src/drivers/Onboard.h}, \texttt{Onboard.cpp} \hfill
\textbf{継承:} \texttt{IModule} \hfill
\phasebadge{Output Phase}

\vspace{2mm}\hrule\vspace{2mm}

\apisection{概要}

GR-PEACH ボード上のRGB LED (3個) + ユーザーLED (1個) およびユーザースイッチ (1個) を制御する。
LED はラッチ方式、スイッチは即時読み取り。

\apisection{機能一覧}

\begin{itemize}[nosep]
  \item RGB LED 3個 + ユーザーLED 1個の制御
  \item ラッチ方式によるLED出力
  \item ユーザースイッチの即時読み取り
\end{itemize}

\apisection{内部変数}

\begin{tabularx}{\textwidth}{l l l X}
\toprule
\textbf{変数名} & \textbf{型} & \textbf{初期値} & \textbf{説明} \\
\midrule
\texttt{\_ledState[4]} & \texttt{int[4]} & \texttt{\{0,0,0,0\}} & 各LED状態 (0:消灯, 1:点灯) \\
\bottomrule
\end{tabularx}

\apisection{関数ドキュメント}

\begin{funcblock}{init()}
\texttt{void init(void)}

\vspace{1.5mm}
GPIO初期化を行う。全LEDを消灯し、スイッチを入力モードに設定する。
\texttt{PIBC6} の入力バッファ有効化を含む。
\end{funcblock}

\begin{funcblock}{update()}
\texttt{void update(void)}

\vspace{1.5mm}
内部変数のLED状態をGPIOへ反映する。内部で \texttt{writeLed()} を各LED IDに対して呼び出す。
\end{funcblock}

\begin{funcblock}{setLed(id, val)}
\texttt{void setLed(int id, int val)}

\vspace{1.5mm}
指定LEDの状態を内部変数に保存する(ラッチ方式)。

\vspace{1.5mm}
\begin{itemize}[nosep, leftmargin=*]
  \item \texttt{id} (\texttt{int}) --- LED番号 (0:Red, 1:Green, 2:Blue, 3:User)
  \item \texttt{val} (\texttt{int}) --- 0:消灯, 1:点灯
\end{itemize}
\end{funcblock}

\begin{funcblock}{sw()}
\texttt{int sw(void)}

\vspace{1.5mm}
ユーザースイッチの状態を即時読み取りで返す。ラッチ方式\textbf{ではなく}、呼び出し時にGPIOを直接読む。
内部で \texttt{readSw()} を呼び出す。

\vspace{1.5mm}
\textbf{戻り値:} \texttt{int} --- 1:押下, 0:解放
\end{funcblock}

\begin{funcblock}{writeLed(id, val) \normalfont\small\color{gray}[内部]}
\texttt{void writeLed(int id, int val)}

\vspace{1.5mm}
指定LEDのGPIOを直接操作する。RGB LEDはLow Active、ユーザーLEDはHigh Active。

\vspace{1.5mm}
\begin{itemize}[nosep, leftmargin=*]
  \item \texttt{id} (\texttt{int}) --- LED番号
  \item \texttt{val} (\texttt{int}) --- 論理的な点灯/消灯状態
\end{itemize}
\end{funcblock}

\begin{funcblock}{readSw() \normalfont\small\color{gray}[内部]}
\texttt{int readSw(void)}

\vspace{1.5mm}
SW用GPIO (P6\_0) を読み取り、論理値に変換して返す。Low = 押下。

\vspace{1.5mm}
\textbf{戻り値:} \texttt{int} --- 1:押下, 0:解放
\end{funcblock}

\apisection{ハードウェアマッピング}

\begin{tabularx}{\textwidth}{l l l l X}
\toprule
\textbf{デバイス} & \textbf{ID} & \textbf{ピン} & \textbf{論理} & \textbf{説明} \\
\midrule
LED Red    & 0 & P6\_13 & Low Active  & 赤色 LED \\
LED Green  & 1 & P6\_14 & Low Active  & 緑色 LED \\
LED Blue   & 2 & P6\_15 & Low Active  & 青色 LED \\
LED User   & 3 & P6\_12 & High Active & ユーザー LED \\
SW User    & -- & P6\_0   & Low = 押下   & プルアップ、GND 接続 \\
\bottomrule
\end{tabularx}

\begin{cautionbox}[GPIO注意事項]
\begin{itemize}[nosep]
  \item RGB LED は \textbf{Low Active}(GPIO Low で点灯)
  \item スイッチ読み取りには \texttt{PIBC6} の入力バッファ有効化が必須
  \item \texttt{sw()} はラッチ方式\textbf{ではなく}、呼び出し時に直接GPIOを読む
\end{itemize}
\end{cautionbox}

\end{apibox}

%% ── 7.8 Serial ──
\subsection{Serial クラス}\label{sec:api-serial}

\begin{apibox}{Serial --- シリアル通信ドライバ}

\textbf{ファイル:} \texttt{src/drivers/Serial.h}, \texttt{Serial.cpp} \hfill
\textbf{継承:} \texttt{IModule} \hfill
\phasebadge{Output Phase}

\vspace{2mm}\hrule\vspace{2mm}

\apisection{概要}

SCIF (Serial Communication Interface with FIFO) を使用してデバッグ用シリアル通信を行う。
送信はバッファリングし、\texttt{update()} で順次送出する。

\apisection{機能一覧}

\begin{itemize}[nosep]
  \item SCIF によるシリアル通信
  \item リングバッファによる送信バッファリング
  \item 書式付き出力 (\texttt{printf})
  \item \texttt{update()} による非ブロッキング送出
\end{itemize}

\apisection{内部定数}

\begin{tabularx}{\textwidth}{l l X}
\toprule
\textbf{定数名} & \textbf{値} & \textbf{説明} \\
\midrule
\texttt{BAUD\_RATE}   & 115200 & ボーレート \\
\texttt{TX\_BUF\_SIZE} & 256    & 送信バッファサイズ (bytes) \\
\bottomrule
\end{tabularx}

\apisection{内部変数}

\begin{tabularx}{\textwidth}{l l l X}
\toprule
\textbf{変数名} & \textbf{型} & \textbf{初期値} & \textbf{説明} \\
\midrule
\texttt{\_txBuf[]}  & \texttt{char[]}    & \texttt{0} & 送信リングバッファ \\
\texttt{\_txHead}   & \texttt{uint16\_t} & \texttt{0} & 書込みポインタ \\
\texttt{\_txTail}   & \texttt{uint16\_t} & \texttt{0} & 読出しポインタ \\
\bottomrule
\end{tabularx}

\apisection{関数ドキュメント}

\begin{funcblock}{init()}
\texttt{void init(void)}

\vspace{1.5mm}
SCIFの初期化・ボーレート設定を行う。送信バッファをクリアする。
\end{funcblock}

\begin{funcblock}{update()}
\texttt{void update(void)}

\vspace{1.5mm}
送信バッファの内容をFIFOへ順次送出する。FIFO空き状況に応じて部分送出する。
内部で \texttt{flushTx()} を呼び出す。
\end{funcblock}

\begin{funcblock}{print(str)}
\texttt{void print(const char* str)}

\vspace{1.5mm}
文字列を送信バッファに追加する。バッファ満杯時は古いデータを上書き。

\vspace{1.5mm}
\textbf{引数:} \texttt{str} (\texttt{const char*}) --- 送信文字列
\end{funcblock}

\begin{funcblock}{printf(fmt, ...)}
\texttt{void printf(const char* fmt, ...)}

\vspace{1.5mm}
書式付き文字列を送信バッファに追加する。\texttt{vsnprintf} を内部で使用。

\vspace{1.5mm}
\textbf{引数:} \texttt{fmt} (\texttt{const char*}) --- 書式文字列(\texttt{printf} 互換)
\end{funcblock}

\begin{funcblock}{putChar(c) \normalfont\small\color{gray}[内部]}
\texttt{void putChar(char c)} --- 1文字をFIFOへ書き込む。
\end{funcblock}

\begin{funcblock}{isTxReady() \normalfont\small\color{gray}[内部]}
\texttt{bool isTxReady(void)} --- 送信FIFOに空きがあるかを返す。
\end{funcblock}

\begin{funcblock}{flushTx() \normalfont\small\color{gray}[内部]}
\texttt{void flushTx(void)} --- バッファ内容を可能な限り一括送出する。
\end{funcblock}

\end{apibox}

%% ── 7.9 SDCard ──
\subsection{SDCard クラス}\label{sec:api-sdcard}

\begin{apibox}{SDCard --- SDカードログドライバ}

\textbf{ファイル:} \texttt{src/drivers/SDCard.h}, \texttt{SDCard.cpp} \hfill
\textbf{継承:} \texttt{IModule} \hfill
\phasebadge{Output Phase}

\vspace{2mm}\hrule\vspace{2mm}

\apisection{概要}

SPI経由でSDカードへログデータを書き込む。
データはバッファリングし、\texttt{update()} でブロック単位で書き出す。

\apisection{機能一覧}

\begin{itemize}[nosep]
  \item SPI 経由の SD カードアクセス
  \item バッファリングによる効率的なブロック書込み
  \item ログ文字列の追記
  \item 明示的フラッシュ・クローズ操作
\end{itemize}

\apisection{内部定数}

\begin{tabularx}{\textwidth}{l l X}
\toprule
\textbf{定数名} & \textbf{値} & \textbf{説明} \\
\midrule
\texttt{BLOCK\_SIZE}     & 512  & SDブロックサイズ (bytes) \\
\texttt{WRITE\_BUF\_SIZE} & 1024 & 書込みバッファサイズ (bytes) \\
\bottomrule
\end{tabularx}

\apisection{内部変数}

\begin{tabularx}{\textwidth}{l l l X}
\toprule
\textbf{変数名} & \textbf{型} & \textbf{初期値} & \textbf{説明} \\
\midrule
\texttt{\_writeBuf[]} & \texttt{uint8\_t[]} & \texttt{0}     & 書込みバッファ \\
\texttt{\_bufPos}     & \texttt{uint16\_t}  & \texttt{0}     & バッファ書込み位置 \\
\texttt{\_isReady}    & \texttt{bool}       & \texttt{false} & SD初期化完了フラグ \\
\texttt{\_blockAddr}  & \texttt{uint32\_t}  & \texttt{0}     & 現在の書込みブロックアドレス \\
\bottomrule
\end{tabularx}

\apisection{関数ドキュメント}

\begin{funcblock}{init()}
\texttt{void init(void)}

\vspace{1.5mm}
SPI初期化およびSDカードの初期化シーケンス (CMD0→CMD8→ACMD41) を実行する。
初期化成功で \texttt{\_isReady} を \texttt{true} に設定。
\end{funcblock}

\begin{funcblock}{update()}
\texttt{void update(void)}

\vspace{1.5mm}
バッファが1ブロック分 (512B) 以上溜まっている場合、SDカードへブロック単位で書き出す。
内部で \texttt{writeBlock()} を呼び出す。
\end{funcblock}

\begin{funcblock}{log(str)}
\texttt{void log(const char* str)}

\vspace{1.5mm}
ログ文字列をバッファに追加する。

\vspace{1.5mm}
\textbf{引数:} \texttt{str} (\texttt{const char*}) --- ログ文字列
\end{funcblock}

\begin{funcblock}{flush()}
\texttt{void flush(void)}

\vspace{1.5mm}
バッファ内容を即時書き出す。残りデータが512B未満でもパディングして書き出す。
\end{funcblock}

\begin{funcblock}{close()}
\texttt{void close(void)}

\vspace{1.5mm}
\texttt{flush()} を呼び出した後、ファイル終端処理を行う。
SD通信を終了し、\texttt{\_isReady} を \texttt{false} に設定する。
\end{funcblock}

\begin{funcblock}{spiWrite(byte) \normalfont\small\color{gray}[内部]}
\texttt{uint8\_t spiWrite(uint8\_t byte)}

\vspace{1.5mm}
SPI 1バイト送受信を行う。送信と同時に受信データを返す。

\vspace{1.5mm}
\textbf{引数:} \texttt{byte} (\texttt{uint8\_t}) --- 送信バイト

\textbf{戻り値:} \texttt{uint8\_t} --- 受信バイト
\end{funcblock}

\begin{funcblock}{sendCmd(cmd, arg) \normalfont\small\color{gray}[内部]}
\texttt{uint8\_t sendCmd(uint8\_t cmd, uint32\_t arg)}

\vspace{1.5mm}
SDカードコマンドを送信し、レスポンスを返す。

\vspace{1.5mm}
\begin{itemize}[nosep, leftmargin=*]
  \item \texttt{cmd} (\texttt{uint8\_t}) --- SDコマンド番号
  \item \texttt{arg} (\texttt{uint32\_t}) --- コマンド引数
\end{itemize}

\textbf{戻り値:} \texttt{uint8\_t} --- R1レスポンス
\end{funcblock}

\begin{funcblock}{writeBlock(data) \normalfont\small\color{gray}[内部]}
\texttt{bool writeBlock(const uint8\_t* data)}

\vspace{1.5mm}
512Bブロックをアドレス \texttt{\_blockAddr} に書き込む。成功で \texttt{\_blockAddr} をインクリメント。

\vspace{1.5mm}
\textbf{引数:} \texttt{data} (\texttt{const uint8\_t*}) --- 書込みデータ (512B)

\textbf{戻り値:} \texttt{bool} --- 書込み成功/失敗
\end{funcblock}

\begin{funcblock}{waitReady() \normalfont\small\color{gray}[内部]}
\texttt{bool waitReady(void)}

\vspace{1.5mm}
SDカードのビジー状態が解除されるまで待機する。タイムアウト時は \texttt{false} を返す。

\vspace{1.5mm}
\textbf{戻り値:} \texttt{bool} --- ビジー解除成功/タイムアウト
\end{funcblock}

\end{apibox}


%% ============================================================
%% 8. 実行モデル詳細
%% ============================================================
\section{実行モデル詳細}\label{sec:execution}

\subsection{割り込みハンドラ}

割り込みハンドラ関数内で各モジュールの \texttt{update()} を呼び出す。
\texttt{Pattern} 処理はクラス化せず、ハンドラ内にベタ書き(または関数化)する。

\begin{lstlisting}[caption={割り込みハンドラ実装例},label=lst:handler]
// グローバルインスタンス
static Onboard onboard;
static Motor motor;
static Servo servo;

void interrupt_handler() {
    // --- Input Phase ---
    // encoder.read();
    // camera.read();

    // --- Logic Phase (Pattern) ---
    int sw = onboard.sw();
    if (sw) {
        motor.setPower(0.2f, 0.2f);  // PWM 20%
        servo.setAngle(10.0f);       // 10度
        onboard.setLed(0, 1);        // Red ON
        onboard.setLed(1, 1);        // Green ON
        onboard.setLed(2, 1);        // Blue ON
    } else {
        motor.setPower(0.0f, 0.0f);
        servo.setAngle(0.0f);
        onboard.setLed(0, 1);        // Red ON
        onboard.setLed(1, 0);        // Green OFF
        onboard.setLed(2, 0);        // Blue OFF
    }

    // --- Output Phase ---
    motor.update();
    servo.update();
    onboard.update();
    // serial.update();
    // sd.update();
}
\end{lstlisting}

\subsection{ラッチ方式ドライバの動作原理}

\begin{figure}[H]
\centering
\begin{tikzpicture}[
  box/.style={draw=headerblue, fill=headerblue!8, rounded corners=2pt,
              minimum width=32mm, minimum height=12mm, font=\small, align=center},
  data/.style={draw=accentblue, fill=accentblue!10, rounded corners=2pt,
               minimum width=25mm, minimum height=9mm, font=\small\ttfamily, align=center},
  arr/.style={-{Stealth[length=3mm]}, thick, color=accentblue},
  lbl/.style={font=\scriptsize\color{gray}, above, midway},
]

\node[box] (logic) {Logic Phase\\(Pattern)};
\node[data, right=15mm of logic] (var) {\_pwmLeft\\\_pwmRight};
\node[box, right=15mm of var] (update) {update()};
\node[data, right=15mm of update] (hw) {MTU2\\レジスタ};

\draw[arr] (logic) -- node[lbl]{setPower()} (var);
\draw[arr] (var) -- node[lbl]{参照} (update);
\draw[arr] (update) -- node[lbl]{書込み} (hw);

\end{tikzpicture}
\caption{ラッチ方式ドライバのデータフロー}
\label{fig:latch}
\end{figure}

\end{document}
